\section{Eksperymenty}
W naszym projekcie przeprowadziliśmy walidacji krzyżowej dla parametru k równego 10 dla drzewa utworzonego przez algorytm ID3 oraz dla drzewa utworzonego prze C4.5. Test został powtórzony 100 razy, a uśrednione wyniki zostały przedstawione poniżej.

\subsection{Wyniki}



\begin{table}[H]
    \centering
    \begin{tabular}{|c|c|}
    \hline
    Algorytm                & Dopasowanie         \\ \hline
    ID3                     & 81,5\%              \\ \hline
    C4.5                    & 81,4\%              \\ \hline
    \end{tabular}
    \caption{Wyniki walidacji krzyżowej dla akceptorów}
    \label{tab:crossing}
\end{table}

\begin{table}[H]
    \centering
    \begin{tabular}{|c|c|}
    \hline
    Algorytm                & Dopasowanie         \\ \hline
    ID3                     & 83,2\%              \\ \hline
    C4.5                    & 81,7\%              \\ \hline
    \end{tabular}
    \caption{Wyniki walidacji krzyżowej dla donorów}
    \label{tab:crossing}
\end{table}

Z wyników walidacji wynika, że drzewo przed zastosowaniem algorytmu C4.5 daje lepszą dokładność wyników niż po. Algorytm C4.5 przyśpiesza działanie programu na utworzonym drzewie oraz pozwala uniknąć nadmiernego dopasowania. Wyniki świadczą o tym, że w naszym przypadku zjawisko to nie występowało na tyle często by polepszyć końcowy wynik, a wręcz przeciwnie.

